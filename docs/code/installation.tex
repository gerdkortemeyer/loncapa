\chapter{Installation}\index{installation}
This is the description of the installation for {\it code developers}, it is not yet how the system will be installed eventually in production.
\section{Linux}\index{Linux}\index{CentOS}
LON-CAPA runs on Linux. Since the target eventually is a server, enterprise editions of Linux are recommended, and in particular CentOS. For CentOS, a ``minimal desktop'' installation is recommended.

For developers, a virtual machine is good enough: something like two cores, four GB RAM, and 40 GB disk will do fine.
\section{Certificate IP/DNS considerations}\index{certificates}
LON-CAPA is a networked system, and all inter-server communication is secured via certificates. It is important that the machine has a stable IP address and DNS. This works fine for simple virtual machines, where this address is {\tt localhost:localdomain}, and the download includes test certificates for this basic configuration.

When using a real server setup building an actual cluster or using a real server, a customized certificate from the LON-CAPA certificate authority\index{certificate authority} is required. Gerd Kortemeyer (for now) can make those certificates, corresponding to {\tt LONCAPA.crt}.
\section{Downloading}
\begin{enumerate}
\item get a username at Github, notify Gerd Kortemeyer (for now) to be listed as collaborator\index{Github}
\item after that, get {\tt https://github.com/gerdkortemeyer/loncapa}
\end{enumerate}
\section{Install LON-CAPA}
\begin{itemize}
\item in subdirectory {\tt install}, use {\tt install\_packages.sh} (do this on a fast connection, there are a ton of libraries, etc)
\item in subdirectory {\tt testcerts}, use {\tt install\_test\_certs.sh}
\item if not {\tt localhost:localdomain}, now copy customized server certificates into {\tt /home/loncapa/certs} --- please do not overwrite the sandbox certificates in the repository
\item if not {\tt localhost:localdomain}, the cluster table (see Section~\ref{clustertable} on page~\pageref{clustertable}) and (possibly) the server name in the Apache configuration {\tt httpd.conf} need adjusting\index{cluster table} --- please do not overwrite the sandbox configurations in the repository
\item back in {\tt install}, use {\tt install.sh}
\item see if the whole thing starts. Then call {\tt http://localhost/test} - that should make an initial user ``zaphod'' with password ``zaphodB'' (for now)
\end{itemize}
\section{If things don't work}
If things don't work, the following log-files might be helpful:\index{log files}
\begin{itemize}
\item Apache server error logs {\tt /etc/httpd/logs/error\_log} and {\tt /etc/httpd/logs/ssl\_error\_log} (or equivalent in distributions other than CentOS)
\item LON-CAPA log files {\tt /home/loncapa/logs/errors.log} and {\tt /home/loncapa/logs/warnings.log}
\end{itemize}
